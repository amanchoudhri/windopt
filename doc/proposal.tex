\documentclass[12pt]{article}

\usepackage[T1]{fontenc}
\usepackage[full]{textcomp}
\usepackage{newtxtext}
\usepackage{cabin} % sans serif
\usepackage[varqu,varl]{inconsolata} % sans serif typewriter
\usepackage[final,expansion=alltext]{microtype}
\usepackage[english]{babel}
\usepackage{amsmath}
\usepackage[bigdelims,vvarbb]{newtxmath} % bb from STIX

% geometry of the page

\usepackage[top=1in,
            bottom=1in,
            left=1in,
            right=1in]{geometry}

% paragraph spacing

\setlength{\parindent}{0pt}
\setlength{\parskip}{2ex plus 0.4ex minus 0.2ex}

% useful packages

\usepackage{epsfig}
\usepackage{url}
\usepackage{bm}
\usepackage{blindtext}


\begin{document}

\begin{flushleft}
\textbf{Project Proposal: Multi-fidelity BO for Wind Farm Layout Optimization} \\
Aman Choudhri (ac4972) \\
\today
\end{flushleft}

\vspace{0.1in}

Wind farms are among the cleanest sources of energy, given their low greenhouse
gas emissions and minimal water consumption \cite{chughWindFarmLayout2022}. One
practical challenge with constructing and deploying wind turbines in practice,
however, comes from the fact that turbines placed in nearby locations will
affect each others' performance by altering the strength and turbulence of the
wind. After wind passes through a turbine, it becomes weaker and more
turbulent, which can lead to less power generation and higher fatigue on
downwind turbines \cite{hellanBayesianOptimisationClimate2023}. The zone after a turbine
in which the wind flow is disrupted is known as the \emph{wake}. Optimizing the
layout of turbines given their wakes, then, is essential to the feasibility and performance of
wind farm energy generation in practice.

There are a variety of strategies to calculate or simulate the wake pattern and
power performance of a wind farm given the layout of its turbines. The gold
standard for wake prediction arises from an expensive computational fluid
dynamics technique known as large-eddy simulation (LES)
\cite{niayifarAnalyticalModelingWind2016}. Given the cost of running such
simulations, many authors have proposed cheaper, analytical approximations. One
commonly used strategy is the Jensen model
\cite{jimenezApplicationTechniqueCharacterize2010}, for example. But as described
in \cite{bliekEXPObenchBenchmarkingSurrogatebased2023}, even one run of the cheaper
Jensen model calculation may take up to 15 seconds on a CPU.

The high computational cost of even the cheap analytic approximations has made wind
farm layout optimization a suitable candidate for Bayesian Optimization (BO).
In \cite{chughWindFarmLayout2022}, for example, the authors introduce a
set-valued Gaussian Process surrogate model and apply expected hypervolume
improvement to learn layout configurations along the performance/cost Pareto
frontier. More recent papers \cite{bempedelisDatadrivenOptimisationWind2024}
\cite{moleMultiFidelityBayesianOptimisation2024} incorporate the
higher-fidelity expensive LES model into optimization routines, but focusing primarily
on the related problem of optimizing the \emph{angles} of turbines in fixed locations.

In this project, I propose a BO approach that takes into account multiple
approximation schemes along with the expensive LES procedure. Specifically, I
aim to apply asynchronous multi-fidelity batch Bayesian optimization, as
described in \cite{folchCombiningMultiFidelityModelling2023}, to the wind farm
layout optimization problem. The setup of that paper is as follows.

Multi-fidelity optimization is conceptually simple. Assume we have $1, \ldots,
M$ auxiliary functions $f^{(m)}$ that approximate the function we care about,
$f$, with varying fidelities. Each auxiliary function has a fixed known
evaluation cost, $C^{(m)}$. Either using independent GP surrogates or one joint
multi-output GP (MOGP), we learn a model of the objective values which we then
use in BO. At each step of the optimization procedure, then, we simply select
an input location $x$ and a fidelity $m$ to obtain an observation $f^{(m)}(x)$.

In \cite{folchCombiningMultiFidelityModelling2023}, the authors note that
higher fidelity observations may take \emph{far longer} to obtain. We may want
to continue querying from lower-fidelity observations \emph{while} we wait for
the higher fidelity observation. 

They posit a fixed computational bandwidth
$\Lambda$, and assign each fidelity a batch space parameter $\lambda^{(m)}$
that indicates how much of the compute bandwidth is taken up at one time. At
each timestep, then, we may obtain values from the previous timestep with
cheaper observations, or many timesteps before, in the case of more expensive
observations. To solve this problem, they apply Thompson Sampling to perform
batch BO, and use either an upper confidence bound variant or the expected
information gain to select the fidelity of the following observation.

In the context of the wind farm problem, I propose to use two auxiliary
functions. The Jensen approximate analytical model as the low-fidelity
auxiliary, and the Winc3D large-eddy simulation as the expensive, high-fidelity
auxiliary \cite{deskosWInc3DNovelFramework2020}, following
\cite{moleMultiFidelityBayesianOptimisation2024}. The input space will be
``configuration sets'', following \cite{chughWindFarmLayout2022}, which are a
collection of coordinate locations for the turbines in a layout.

\newpage

\bibliographystyle{plain}
\bibliography{references.bib}

\end{document}

%%% Local Variables:
%%% mode: latex
%%% TeX-master: t
%%% End:
% fonts
