\documentclass[12pt]{article}

\usepackage[T1]{fontenc}
\usepackage[full]{textcomp}
\usepackage{newtxtext}
\usepackage{cabin} % sans serif
\usepackage[varqu,varl]{inconsolata} % sans serif typewriter
\usepackage[final,expansion=alltext]{microtype}
\usepackage[english]{babel}
\usepackage{amsmath}
\usepackage[bigdelims,vvarbb]{newtxmath} % bb from STIX

% geometry of the page

\usepackage[top=1in,
            bottom=1in,
            left=1in,
            right=1in]{geometry}

% paragraph spacing

\setlength{\parindent}{0pt}
\setlength{\parskip}{2ex plus 0.4ex minus 0.2ex}

% useful packages

\usepackage{epsfig}
\usepackage{url}
\usepackage{bm}
\usepackage{blindtext}


\begin{document}

\begin{flushleft}
\textbf{Plan: Multi-fidelity BO for Wind Farm Layout Optimization} \\
Aman Choudhri (ac4972) \\
\today
\end{flushleft}


\section{Setup}
\subsection{Notation}
Say we have a function $f_\text{LES}^{(T)} (X, \theta, v, \phi)$ to simulate
the power output from an $N$-turbine wind farm over a $T$-minute duration,
taking parameters:
\begin{itemize}
    \item $X \in \mathbb{R}^{N \times 2}$, a set of coordinate pairs representing the turbine locations,
    \item $\theta \in \mathbb{R}^{N}$, a set of angles representing the
        orientations of each turbine, in an appropriately defined coordinate system,
    \item $v \in \mathbb{R}_{> 0}$, the incoming wind speed to the site,
    \item $\phi \in \mathbb{R}$, the incoming wind angle to the site, again defined in
        an appropriate coordinate system.
\end{itemize}

Following \cite{bempedelisDatadrivenOptimisationWind2024}, we'll assume that temporal windows
larger than 2 hours, or $T \geq 120$, simply return to us the true power performance. Write
this ``true'' power performance as $f_\text{LES}$, so
\[
    f_\text{LES} \coloneq f_\text{LES}^{(T)}, \quad \text{for \ } T \geq 120
.\]

Since evaluations of $f_\text{LES}$ are expensive, we also have access to a
variety of cheap analytic approximations, $f_{\text{approx}}^{(m)}$, for $m =
1,\ldots , M$. We'd like to place minimal assumptions on the correctness of
these low-fidelity observations.

Finally, we might also be interested in obtaining observations that are
higher-quality than $f^{(m)}_\text{approx}$ but cheaper than $f_\text{LES}$. To
get these, we can run shorter simulations $f^{(T)}_\text{LES}$ for smaller $T$.
It seems reasonable to model these fidelities as unbiased relative to
$f_\text{LES}$, but with noise levels dependent on $T$.

\subsection{Wind Farm Optimization}
In the most common case, we want to maximize average-case performance under some historical
distribution of wind speeds and directions, $p(v, \phi)$:
\begin{align*}
    \text{maximize}_{X, \theta}& \quad \mathbb{E}_{p(v, \phi)}[f_\text{LES}(X, \theta, v, \phi)] \\
    \text{s.t.}& \quad g(X) > c
,\end{align*}
where $g(X) > c$ is a feasibility constraint, bounding below by $c$ the
distance between each turbine in configuration $X$.

We might also consider fixing a condition $(v, \phi)$ ahead of time and
optimizing for that condition only. This condition might represent a best or
worst-case scenario.


\section{Methodology}

\subsection{Large Eddy Simulations}
There are several open-source large eddy simulation software packages available online.

The package \texttt{WInc3D} \cite{deskosWInc3DNovelFramework2020} has been used
in \cite{bempedelisDatadrivenOptimisationWind2024,
moleMultiFidelityBayesianOptimisation2024}. A newer package, \texttt{TOSCA}
\cite{stipaTOSCAOpensourceFinitevolume2024}, was released four years later in
2024, with more of a focus on modeling ``gravity waves.'' It does seem like \texttt{WInc3D}
is the smart choice here, though, since both BO papers I've looked at use it.

\subsection{Analytic Wake Models}

Analytic wake models are often implemented in the open-source package
\texttt{FLORIS} \cite{mudafortNRELFlorisV4212024}.

The Gauss-curl hybrid (GCH) model \cite{niayifarAnalyticalModelingWind2016}
seems to be the standard model for this application, being used in
\cite{moleMultiFidelityBayesianOptimisation2024,
bempedelisDatadrivenOptimisationWind2024}.

\subsection{Optimization Procedure}

Asynchronous batch BO with multiple fidelities: \cite{folchCombiningMultiFidelityModelling2023}.
This paper would work for everything, except anything about constituent evaluations.

\section{Experiments}

\subsection{Proof of Concept}

\subsection{Horns Rev}

This is a large offshore wind farm, consisting of 80 turbines in an 8x10 grid with a spacing
between each turbine of $7D$.

In \cite{moleMultiFidelityBayesianOptimisation2024}, the authors optimized the
yaw angles of each row of turbines. They used the following configurations:
\begin{itemize}
    \item LES configuration:
        \begin{itemize}
            \item Uniform grid of size $74D \times 500m \times 7D$.
            \item Grid spacing of $D / 10$ in each spatial direction.
            \item Power output averaged over 2hrs, timestep of 0.2 seconds.
            \item Third-order Adams-Bashforth time advancement.
            \item Grid scale filter coefficient of $\alpha_\text{filter} = 0.49$.
            \item ABL parameters:
                \begin{itemize}
                    \item Friction velocity $u^* = 0.442 m / s$
                    \item Boundary layer height of $\delta = 504m$
                    \item Roughness length $z_{0} = 0.05m$
                \end{itemize}
        \end{itemize}
    \item BO configuration:
        \begin{itemize}
            \item Latin hypercube sampling with 100 low-fidelity and 12
                high-fidelity configurations.
            \item UCB acquisition function
            \item Multi-fidelity GP using ``NARGP'' models.
        \end{itemize}
\end{itemize}

I have essentially four novel-ish ideas or contributions. Here's a draft motivation.

The cost of wind farm energy depends on the annual energy production (AEP) metric, which
necessitates studying wind farm performance in a variety of wind speed and direction conditions,
not just the optimal environment. Calculating the performance of a given wind farm layout in
multiple conditions, however, requires re-running the expensive large eddy simulations for
each wind speed and direction. Optimizing wind farm configurations for a variety of wind farm
conditions, then, is computationally prohibitive.

In this paper, we take advantage of the ``grey-box'' Bayesian optimization literature on
``constituent evaluations'' to address this problem and allow for the optimization of
wind farm configurations with average-case performance as an objective.

To further reduce the computational burden of running multiple high-fidelity LESs,
we introduce a novel (to our knowledge) multiple-fidelity formulation of a large-eddy simulation
that allows an optimization routine to adaptively select the temporal window over which
wind farm power production is averaged and calculated. This gives the routine the option of obtaining
cheaper, noisier evaluations from a large eddy simulation; such an option may be desirable,
especially when trying to quickly understand the average-case performance of a given configuration,
if more precision than an analytical wake model approximation is required.

We expect the benefits of considering average-case performance to be most
noticeable when we are jointly optimization wind farm layout and orientation,
since the performance of a wake-steered wind farm is likely highly dependent on
the incoming wind direction. To simplify the computational burden of the
higher-dimensional optimization problem, we'll use trust regions.

\subsection{Extension of \cite{moleMultiFidelityBayesianOptimisation2024}}
In \cite{moleMultiFidelityBayesianOptimisation2024}, the authors solve the
wake-steering problem in a multi-fidelity manner. They use a GCH model as
low-fidelity option, and a \texttt{WInc3D} simulation as the high-fidelity option.

Questions relative to this paper:
\begin{itemize}
    \item Is the NARGP prior really necessary?
    \item Do we save any computation by using a shorter temporal averaging window?
    \item Do we lose any performance in doing so?
\end{itemize}

\subsection{Compare to \cite{bempedelisDatadrivenOptimisationWind2024}}
\subsection{Whole Boy}

Jointly solving wake steering and micro-siting.

%
% There are a variety of strategies to calculate or simulate the wake pattern and
% power performance of a wind farm given the layout of its turbines. The gold
% standard for wake prediction arises from an expensive computational fluid
% dynamics technique known as large-eddy simulation (LES)
% \cite{niayifarAnalyticalModelingWind2016}. Given the cost of running such
% simulations, many authors have proposed cheaper, analytical approximations. One
% commonly used strategy is the Jensen model
% \cite{jimenezApplicationTechniqueCharacterize2010}, for example. But as described
% in \cite{bliekEXPObenchBenchmarkingSurrogatebased2023}, even one run of the cheaper
% Jensen model calculation may take up to 15 seconds on a CPU.
%
% The high computational cost of even the cheap analytic approximations has made wind
% farm layout optimization a suitable candidate for Bayesian Optimization (BO).
% In \cite{chughWindFarmLayout2022}, for example, the authors introduce a
% set-valued Gaussian Process surrogate model and apply expected hypervolume
% improvement to learn layout configurations along the performance/cost Pareto
% frontier. More recent papers \cite{bempedelisDatadrivenOptimisationWind2024}
% \cite{moleMultiFidelityBayesianOptimisation2024} incorporate the
% higher-fidelity expensive LES model into optimization routines, but focusing primarily
% on the related problem of optimizing the \emph{angles} of turbines in fixed locations.
%
% In this project, I propose a BO approach that takes into account multiple
% approximation schemes along with the expensive LES procedure. Specifically, I
% aim to apply asynchronous multi-fidelity batch Bayesian optimization, as
% described in \cite{folchCombiningMultiFidelityModelling2023}, to the wind farm
% layout optimization problem. The setup of that paper is as follows.
%
% Multi-fidelity optimization is conceptually simple. Assume we have $1, \ldots,
% M$ auxiliary functions $f^{(m)}$ that approximate the function we care about,
% $f$, with varying fidelities. Each auxiliary function has a fixed known
% evaluation cost, $C^{(m)}$. Either using independent GP surrogates or one joint
% multi-output GP (MOGP), we learn a model of the objective values which we then
% use in BO. At each step of the optimization procedure, then, we simply select
% an input location $x$ and a fidelity $m$ to obtain an observation $f^{(m)}(x)$.
%
% In \cite{folchCombiningMultiFidelityModelling2023}, the authors note that
% higher fidelity observations may take \emph{far longer} to obtain. We may want
% to continue querying from lower-fidelity observations \emph{while} we wait for
% the higher fidelity observation. 
%
% They posit a fixed computational bandwidth
% $\Lambda$, and assign each fidelity a batch space parameter $\lambda^{(m)}$
% that indicates how much of the compute bandwidth is taken up at one time. At
% each timestep, then, we may obtain values from the previous timestep with
% cheaper observations, or many timesteps before, in the case of more expensive
% observations. To solve this problem, they apply Thompson Sampling to perform
% batch BO, and use either an upper confidence bound variant or the expected
% information gain to select the fidelity of the following observation.
%
% In the context of the wind farm problem, I propose to use two auxiliary
% functions. The Jensen approximate analytical model as the low-fidelity
% auxiliary, and the Winc3D large-eddy simulation as the expensive, high-fidelity
% auxiliary \cite{deskosWInc3DNovelFramework2020}, following
% \cite{moleMultiFidelityBayesianOptimisation2024}. The input space will be
% ``configuration sets'', following \cite{chughWindFarmLayout2022}, which are a
% collection of coordinate locations for the turbines in a layout.

\newpage

\bibliographystyle{plain}
\bibliography{references.bib}

\end{document}

%%% Local Variables:
%%% mode: latex
%%% TeX-master: t
%%% End:
% fonts
