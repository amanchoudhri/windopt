\documentclass[12pt]{article}

\usepackage[T1]{fontenc}
\usepackage[full]{textcomp}
\usepackage{newtxtext}
\usepackage{cabin} % sans serif
\usepackage[varqu,varl]{inconsolata} % sans serif typewriter
\usepackage[final,expansion=alltext]{microtype}
\usepackage[english]{babel}
\usepackage{amsmath}
\usepackage[bigdelims,vvarbb]{newtxmath} % bb from STIX
\usepackage{hyperref}
\usepackage{enumitem}

\setlist{nolistsep}

% Reference management
\usepackage[style=numeric]{biblatex}
\bibliography{references.bib}

% Custom citation command
\DeclareCiteCommand{\longcite}
  {\usebibmacro{prenote}}
  {%
    % Print author names
    \printnames{author}%
    \addperiod\space
    % Print title in quotes with hyperlink (if DOI or URL exists)
    \iffieldundef{doi}
      {\iffieldundef{url}
        % If no DOI or URL, just print title
        {\mkbibquote{\printfield{title}}}
        % If URL exists, link to it
        {\href{\thefield{url}}{\mkbibquote{\printfield{title}}}}}
      % If DOI exists, link to DOI
      {\href{https://doi.org/\thefield{doi}}{\mkbibquote{\printfield{title}}}}%
    \addperiod\space
    % Print container (journal/conference/book)
    \iffieldundef{journaltitle}
      {\iffieldundef{booktitle}
        {}
        {\printfield{booktitle}}}
      {\printfield{journaltitle}}%
    \addperiod\space
    % Print year
    \printfield{year}%
    \addperiod
  }
  {\multicitedelim}
  {\usebibmacro{postnote}}


% geometry of the page

\usepackage[top=1in,
            bottom=1in,
            left=1in,
            right=1in]{geometry}

% paragraph spacing

\setlength{\parindent}{0pt}
\setlength{\parskip}{2ex plus 0.4ex minus 0.2ex}

% useful packages

\usepackage{epsfig}
\usepackage{url}
\usepackage{bm}
\usepackage{blindtext}

\newcommand{\term}[1]{\emph{#1.}}

\begin{document}

\begin{flushleft}
\textbf{Master Document: Multi-Fidelity BO} \\
Aman Choudhri (ac4972) \\
\today
\end{flushleft}

\tableofcontents
\newpage

\section{Definitions and Thoughts}
\subsection{Useful Definitions}
\term{Large eddy simulation (LES)}
A numerical method to approximately simulate turbulent flows more cheaply than
directly solving the Navier-Stokes equations. The key idea behind LES is to
essentially ignore the smallest length scales, which add significant
computational overhead, via a low-pass filter on the Navier-Stokes equations.
Instead, the small length scales are modeled using ``subgrid-scale (SGS)
models,'' which are usually so-called ``eddy-viscosity models.''

\term{Atmospheric boundary layer (ABL)}
The lowest level of the atmosphere, a region directly influenced by contact
with the Earth's surface. Above the ABL, the wind flows parallel to level
curves of equal pressure, but within in the flow is far more complex. It
appears that there are multiple common ``profiles'' of ABLs used in modeling:
neutral, stable, unstable, convective. Further, it appears that there are two
main approaches for modeling the ABL, either a ``precursor LES'' or a
``synthetic'' approach \cite{bretonSurveyModellingMethods2017,
mehtaLargeEddySimulation2014}.

\term{Gravity wave effects} 

\term{Thermal stratification} 

\term{Reynolds-Averaged Navier-Stokes (RANS) model}
A model for resolving ``wake turbulence'' that is often used in analytical
engineering models. It has many constants which must be tuned empirically using
real aerodynamic wind farm data \cite[pg.2]{mehtaLargeEddySimulation2014}.
Aside from one case known as the ``Reynolds Stress model,'' RANS models
generally make a certain isotropy assumption on the turbulence, rendering these
models unsuitable for anisotropic flows like the atmospheric boundary layer and
turbine wakes \cite[pg.3]{mehtaLargeEddySimulation2014}.

\term{Power controller} 
Seems to be something relating to constant versus variable speed operation. In
\cite{bretonSurveyModellingMethods2017}, the authors describe a modeling
approach with a constant tip-speed ratio as one with no power controller.

\term{Local thrust coefficient, $C_T$}

\term{Blockage effects}
Mentioned in \cite{bempedelisDatadrivenOptimisationWind2024}.

\subsection{Useful Facts}
As of 2014, a ``normal separation for turbines on a modern wind farm'' is
roughly $7-8D$ \cite[pg.4]{mehtaLargeEddySimulation2014}.

The most important eddies to model accurately are those that are comparable
with or larger than the turbine diameter $D$, and ones which are most
responsible for the transport of mass, momentum, and energy
\cite[pg.4]{mehtaLargeEddySimulation2014}.

LES are preferable to engineering models because of their ability to
``capture the transient evolution of turbulent eddies that are most relevant
to wake development and power production'' \cite[pg.9]{mehtaLargeEddySimulation2014}.

Field data is highly noisy due to the ``erratic nature of the atmosphere,''
which may compromise the ability of engineering models to deliver reliable
predictions in tail regimes like ``very large wind farms or inflow angles
involving high number of wake interactions.''
\cite[pg.15]{mehtaLargeEddySimulation2014}.

As the grid resolution of an LES becomes coarser, the fidelity and accuracy of
ABS and SGS modeling becomes more and more important
\cite{mehtaLargeEddySimulation2014}.

Thermal stratification in the ABL means that ``atmospheric gravity waves can be
triggered above the ABL'' \cite{stipaTOSCAOpensourceFinitevolume2024}.

It seems like we need to be very careful with periodic boundary conditions,
because a potential concern is that wakes may be ``re-introduced into the
domain'' \cite[pg.303]{stipaTOSCAOpensourceFinitevolume2024}.

``Actuator disk theory tends to overestimate the power output of wind turbines
at large yaw angles'' \cite[pg.878]{bempedelisDatadrivenOptimisationWind2024},
which actually came from a 2019 Porte-Agel paper. Bempedelis is an advocate for
actuator disk theory, nonetheless—the comment was included only as a nod to a reviewer.

“The normalized velocity deficit in the turbine wakes has been observed to
follow a self-similar Gaussian profile in several experimental and numerical
research studies” \cite[pg.2]{niayifarAnalyticalModelingWind2016}.


\subsection{Working Thoughts and Observations}
It seems like we only care about modeling the far wake? Maybe we place turbines within $6D$
of each other, but that does seem kind of close...

Essentially the closer the turbines are together, the more accurately we want
to model. In section 4.2.2 of \cite{mehtaLargeEddySimulation2014}, the authors
describe a study that used the incredibly-expensive but more accurate
``actuator line'' approach to model turbines since the spacing was roughly
$4.3D$, which ``would require proper modelling of the near wake.''

ABL modeling and potentially ``coupling'' with aeroelastic codes could be
useful for modeling turbine loading in ``off-design conditions like non-neutral
ABLs and gusts'' \cite{mehtaLargeEddySimulation2014}. This could maybe be useful
if we're framing this paper as trying to accurately estimate annual energy production
across a variety of historical wind conditions.

It seems like there is some kind of ``inherent variability of LES,''
\cite{bretonSurveyModellingMethods2017}, which reminds me of the phrase
``time-averaged.'' Is there some kind of literature on the length of time one
runs an LES for to determine a power production for a given wind speed? It seems like
\cite{andersenQuantifyingVariabilityLarge2015} discusses this in detail.

It seems like the idea of ABL modeling is because ``wind turbine wake recovery
and thus power production are greatly influenced by background atmospheric
turbulence'' \cite[pg.302]{stipaTOSCAOpensourceFinitevolume2024}.


FRAMING: Loosely motivated by the results observed in
\cite{bempedelisDatadrivenOptimisationWind2024}, where a wake model
gradient-based optimization routine produced better results than an LES
simulation in 30\% of runs, I'm curious about the idea of framing this paper in
terms of LES checking the results of wake models. Especially since the actual
efficiency predictions from wake models was so drastically different across
FLORIS and LES, again in \cite{bempedelisDatadrivenOptimisationWind2024}

TODO: Let's perform a detailed review of papers comparing the performance of
wake models and LES simulations.

\subsection{Question Log}
\subsubsection{Research Directions}
\begin{enumerate}
    \item Can we model the effects of downstream dynamic loading on turbines
        and their longevities? What have people done in the past? Can we
        formulate this as a multi-objective optimization problem? Or by
        reducing wake effects in general do we solve this inherently?
    \item Can we take advantage of our knowledge about the kinds of eddies that are most
        important to model correctly to specify more carefully the relationship
        between multiple fidelities of wake models? Ie if we know that
        an analytical approximation underestimates wake performance in a certain regime,
        we can get a better sense for how the resulting observation might be biased?
    \item Can we encourage the model to tune different parameters of the LES
        or use different approximations based on the specific input case? Or is
        that way too much effort?
    \item On this note of tuning different parameters, can we take advantage of what we
        know about the physical model to understand the BIAS or NOISE of certain
        observation fidelities in different input regions? I.e. we might expect higher
        noise or a certain bias direction in the LES model based on a function $g(X)$
        defined on our configuration space, which returns the average separation
        between turbines?
    \item On this further note, is it worth also incorporating a RANS fidelity?
        I guess to understand this, my main questions would be: how well
        does RANS perform compared to LES? Like what are the real main disadvantages?
        And what are the real computational advantages?
    \item Can we take advantage of interesting \emph{mixed} approximations like
        what is described in section 4.3 of
        \cite{mehtaLargeEddySimulation2014}, which uses LES to improve simpler
        models? Some keywords to look into here include a ``modified Frandsen's model''
        and an ``LES generated transfer function.''
    \item Is it worth incorporating different rotor dimensions? If I go the TURBO route
        and try to make the case that BO is great in high-dimensional settings, it could
        make sense. It could also add additional complexity and make it more difficult
        to do anything interesting relating to the GP kernel.
    \item Can we use a multi-output GP that gives us the wind speed at each turbine location
        as our model? And then somehow average over the power productions at each location
        in our acquisition function? This way we can maybe more carefully specify the bias
        or variance of a given observation fidelity $f^{(m)}$?
    \item Does it make sense to think about running multiple LESs in parallel
        as well? Can we run one LES and one RANS in parallel? Or should we cap
        it at one and use all available cores for parallelization of the actual LES
        algorithm?
    \item Can we adaptively select the temporal averaging window for a given simulation
        to result in various levels of observation noise?
\end{enumerate}

\subsubsection{General/Empirical Questions}
\begin{enumerate}
    \item What is this concept of ``spanwise'' versus ``streamwise'' spacing?
        Does it refer to different behavior in terms of the spacing of turbines
        in directions parallel to versus perpendicular to the wind inflow
        direction? How do existing BO studies handle this distinction
        constraint-wise, if at all?
    \item Should we use pseudo-spectral or finite-volume approaches? What are the advantages
        and disadvantages of each?
    \item What are the key ideas behind different wake models and their assumptions? Why do
        people use the GCH model \cite{niayifarAnalyticalModelingWind2016}? Does it just empirically
        work very well?
    \item What is the actual variability of the simulations in a 50-turbine setting? Can we get
        away with 10-minutes of wind farm time? Can we bound the variance of the estimations in those contexts?
        Is there a burn-in period?
\end{enumerate}

\section{Literature Review}

\subsection{Bayesian Optimization Methodology}

\longcite{gonzalezBatchBayesianOptimization2016}
\begin{itemize}
    \item \textbf{Overview:} Proposes a new batch BO method based on a
        Lipschitz assumption on the objective function. Uses the GP to infer
        the Lipschitz constant. Greedily selects points within a batch by
        penalizing the acquisition function away from previous points in the
        batch. They base their penalization on the estimated Lipschitz
        constant—specifically, the width of the downweighted region around a
        previously-selected point $x_{t, i}$ is based on how close $x_{t, i}$
        is believed to be to the optimum, which itself is derived from an estimate of
        the Lipschitz constant.
    \item \textbf{Strong points:} 
        \begin{itemize}
            \item Gradient of the penalized acquisition function available in closed form.
        \end{itemize}
    \item \textbf{Weak points:} 
    \begin{itemize}
        \item Assumes that the kernel function $K$ is twice-differentiable, in
            order to estimate the Lipschitz constant.
        \item Assumes that the function is Lipschitz homoskedastic, i.e. that
            the Lipschitz constant of the function is the same everywhere. The
            authors note that the method can be extended to cases where a local Lipschitz
            constant is estimated, but they don't expand on it.
    \end{itemize}
    % \item \textbf{Methodological details:} 
    % \begin{itemize}
    % \end{itemize}
    \item \textbf{Empirical results:}
        % \begin{itemize}
        %     \item Finds a rational quadratic kernel to outperform both a Matern
        %         and squared exponential, where by performance they mean extrapolation performance
        %         on heldout test points.
        %     \item They find that roughly 30\% of the layout designs found by the analytical wake model
        %         outperform their solution using BO and the LES.
        %     \item They also do find that LES outperforms FLORIS in the wake-steering task,
        %         however.
        %     \item FLORIS underpredicts efficiency compared to the true
        %         efficiencies calculated by LES.
        % \end{itemize}
    \item \textbf{Takeaways:}
\end{itemize}


\subsection{Wind Farm Optimization}

\longcite{bempedelisDatadrivenOptimisationWind2024}

\begin{itemize}
    \item \textbf{Overview:} Applies Bayesian optimization with LES and
        analytic wake models to maximize power production in both micro-siting
        and wake steering tasks. Doesn't do anything truly multi-fidelity.
    % \item \textbf{Strong points:} 
    %     \begin{itemize}
    %         \item Allows for \emph{variable numbers of turbines}, with a novel set kernel function
    %         \item Considers \emph{variability in wind speed and direction}. Models power output as an expectation over $p(v, \theta)$, a
    %             joint distribution over wind speed and direction taken from
    %             historical data, estimated using Kernel Density Estimation.
    %     \end{itemize}
    % \item \textbf{Weak points:} 
    % \begin{itemize}
    % \end{itemize}
    \item \textbf{Methodological details:} 
    \begin{itemize}
        \item Smagorinsky SGS model and AD-NR method used, simulated using
            Winc3D package \cite{deskosWInc3DNovelFramework2020}.
        \item Precursor simulations of ``pressure-gradient-driven fully
            developed neutral'' ABLs.
        \item Enforces rotor spacing of $D$. Sets up a $18D \times 18D$ site,
            with $N = 16$ turbines. Picks a $D = 100$m rotor diameter and $h =
            100$m hub height. Assumes a constant local thrust coefficient $C_T'
            = 4 / 3$.
        \item \emph{BO parameters}: No observation noise. Lower confidence
            bound acquisition function. Kernel hyperparameter search and
            acquisition function maximization carried out with L-BFGS
            algorithm. Batching with ``local penalization,'' as described in
            \cite{gonzalezBatchBayesianOptimization2016}.
        \item Evenly-weighted 6-directional wind rose.
        \item Flow data are averaged over a 2.5h period of farm operation (pg.873).
        \item They compare their framework, LES-BO, against a gradient-based
            optimization strategy using the ``Gauss-curl hybrid'' wake model from the FLORIS
            package. The gradient-based optimization strategy relies on what appears to be
            a convergence tolerance threshold, which means it's unclear
            how many evaluations the FLORIS strategy had compared to the LES-BO model.
            A good question to ask the authors.
        \item Simulation costs ranged between 300-900 CPU hours, performed in
            parallel on 128 or 256 cores.
    \end{itemize}
    \item \textbf{Empirical results:}
        \begin{itemize}
            \item Finds a rational quadratic kernel to outperform both a Matern
                and squared exponential, where by performance they mean extrapolation performance
                on heldout test points.
            \item They find that roughly 30\% of the layout designs found by the analytical wake model
                outperform their solution using BO and the LES.
            \item They also do find that LES outperforms FLORIS in the wake-steering task,
                however.
            \item FLORIS underpredicts efficiency compared to the true
                efficiencies calculated by LES.
        \end{itemize}
    \item \textbf{Takeaways:} The performance from the FLORIS wake model simulation is
        maybe not the best sign for the importance of including LES observations. To truly
        understand the cause of the performance difference, however, I want to go into
        their Appendix B and understand how many samples and how many evaluations were used
        for each. Because if the FLORIS approach has far more evaluations available, this
        is maybe not a fair comparison. Could be that optimizing gradients over the analytical
        wake model is likely to result in many local optima.
\end{itemize}

\longcite{chughWindFarmLayout2022}

\begin{itemize}
    \item \textbf{Overview:}
        Applies multi-objective BO with objectives of wind farm power output
        and turbine cost, using expected hypervolume improvement as the
        acquisition function. Formulates the problem as a \emph{set-valued} search space,
        defining a custom kernel over collections of turbine location coordinates and
        enabling search over variable numbers of turbines.
    \item \textbf{Strong points:} 
        \begin{itemize}
            \item Allows for \emph{variable numbers of turbines}, with a novel set kernel function
            \item Considers \emph{variability in wind speed and direction}. Models power output as an expectation over $p(v, \theta)$, a
                joint distribution over wind speed and direction taken from
                historical data, estimated using Kernel Density Estimation.
        \end{itemize}
    \item \textbf{Weak points:} 
    \begin{itemize}
        \item Only considers the Jensen model, with \emph{no large-eddy simulations}.
        \item Discretizes the space of possible turbine locations in the acquisition function to a 20x20 grid.
    \end{itemize}
    \item \textbf{Interesting details:} 
    \begin{itemize}
        \item Maximizes EHVI using a genetic algorithm.
        \item Sets minimum distance between turbines as 3 times rotor diameter.
    \end{itemize}
\end{itemize}


\longcite{hoekPredictingBenefitWake2020}

\begin{itemize}
    \item \textbf{Overview:}
    \item \textbf{Strong points:} 
        \begin{itemize}
            \item Incorporates a \emph{time-varying} wind direction, defining power output as the
                weighted average over the mean direction $\mu$, as well as $\mu \pm 3^\circ$ and $\mu \pm 6^\circ$
        \end{itemize}
    \item \textbf{Weak points:} 
    \begin{itemize}
        \item 
    \end{itemize}
    \item \textbf{Interesting details:} 
    % \begin{itemize}
    % \end{itemize}
\end{itemize}


\subsection{LES Surveys}

\longcite{mehtaLargeEddySimulation2014}
\begin{itemize}
    \item \textbf{Overview:} Summary of various LES implementations to model
        wind farm aerodynamics. Touches on a discussion for how to optimally use LES
        and challenges of such models.
    \item \textbf{Details:} 
    \begin{itemize}
        \item So-called engineering models are simulations using ``basic
            principles of physics and empirically established approximations.''
            They cannot account for phenomena like ``wake meandering...a
            turbine's response to partial wake interaction...'' etc. Generally, the models
            used in practice focus on modeling the \emph{far-wake}, and many resolve turbulence
            using Reynolds-averaged Navier-Stokes models, which have constants that need
            to be empirically specified using real aerodynamic data.
        \item The two key factors of a wake from a turbine are the ``velocity deficit,''
            the reduction in wind velocity due to the energy extracted by the turbine,
            and the ``added turbulence intensity,'' the increase in turbulence
            of the flow within a wake. The velocity deficit reduces the power that downstream
            turbines can generate. The added turbulence intensity increases the ``dynamic loading''
            of downstream turbines, reducing their longevity.
        \item Generally the wake of a turbine is considered as two regions, near and far.
            The near wake is immediately behind the turbine, and the key factor is the
            velocity deficit, which ``attains its maximum value between $1D$ and $2D$,''
            where $D$ is the turbine rotor diameter. In this region, the key
            influencing factors are the design of a turbine and its loading (what?).
            The near wake generally ends between $2D$ and $4D$.
        \item The velocity deficit is generally negligible beyond $10D$ but increased turbulence
            intensity is sensible up to at least $15D$.
        \item Wind turbine and wake flow modeling is a regime well-suited for
            incompressible Navier-Stokes. ``Widely accepted that the
            incompressible form of the NS equations can model the flow around a
            wind turbine and in its wake.'' (page 2)
        \item Wind turbine wakes are high-Reynolds number flows, which make
            direct numerical simulation using Navier-Stokes infeasibly
            expensive. This is because higher Reynolds number flows means
            shorter and shorter length scales of eddies, meaning the overall
            range of possible eddy scales grows larger. (page 3)
        \item The main paradigm for studying turbulence involve some kind of
            statistical averaging, removing certain length and time scales. But
            the problem is underdetermined, known as the ``closure problem,''
            leading to different mathematical models for the flow. This is the origin
            of RANS and LES.
        \item The Reynolds Stress model, which does not make the isotropy
            assumption of most RANS models, ``provides the most reliable
            results in both the near and the far wake,'' among far wake models. It
            does underpredict velocity deficit in the near wake, however (page 4).
        \item Engineering models perform poorly for a single ``inflow angle''
            but do better when averaged across multiple angles. The poor single
            direction performance ``stems from the combination of innacurate turbulence
            and ABL modelling, tuning with limited field data, and the inability to resolve the
            flow in case of multiple turbine-wake or wake-wake interactions.''
        \item Argues that industry needs ``detailed knowledge on the
            performance of wind turbines post-deployment'' in ``off-design''
            regimes, like with gusts or atmospheric stratification.
        \item A common class of ``eddy-viscosity'' SGS models, those derived
            from ``Smagorinsky's model,'' make a certain assumption on the
            alignment between the strain rate and subgrid tensors, which is
            unsupported by numerical simulation data. Other models like the
            ``Scale Similarity Model'' and derivatives like the ``Mixed SSM''
            do not make this assumption.
        \item Wind turbines are usually modeled using \emph{actuator methods} instead
            of direct methods, which are computationally expensive.
        \item ``Energy-conserving schemes'' are flagged as promising, being free from ``numerical dissipation''
            and the requirement to use ``periodic boundaires,'' but are noted as requiring further study.
            I'm curious whether these have been looked into further as of 2024.
        \item Another open-ish question from this paper is the best way to handle boundary conditions
            with the ABL. Not sure what this means, but a keyword is ``Monin-Obukov's approach,''
            which is flagged as being ``unsuitable for LES.'' Other approaches are briefly
            alluded to, but the authors note that more experiments are necessary.
        \item Empirical results:
            \begin{itemize}
                \item The AD-NR actuator disk method is ``suitable and faster'' compared
                    to the more expensive AD-R method (which incorporates ``tangential forces'')
                    if ``one aims to analyze only the power produced and if turbines are separated
                    by at least 5D to 7D'' (pg.4)
            \end{itemize}
    \end{itemize}
\end{itemize}

\longcite{bretonSurveyModellingMethods2017}
\begin{itemize}
    \item \textbf{Overview:} Survey of common schemes for modeling rotor,
        atmospheric conditions, and terrain effects within LES implementations
        as of 2017. Not as focused on the CFD methods. Also summarizes
        experimental research data available for validating LES
        implementations.
    \item \textbf{Details:} 
    \begin{itemize}
        \item Argues that the most practical concern for wind farm simulations is to understand
            the ways that ``wake effects alter turbine loading and power extraction and
            how such effects are influenced by atmosphereic conditions and topography.''
        \item Alternatives to LES models include ``engineering models''
            (analytic wake models?), computations solving the Reynolds-averaged
            Navier-Stokes (RANS) equations, and direct numerical simulations.
            Direct numerical simulations are too expensive to use, they argue.
            And apparently, RANS-based models are ``known to depend on the choice
            of turbulence closure models'' and they can only provide limited
            information about the ``inherently unsteady processes underlying wake phenomena''
            because of their time-averaged formulation.
        \item There are multiple ways to computationally solve the filtered Navier-Stokes equations
            within LESs, including finite difference, finite volume, or pseudo-spectral.
        \item Supports the idea of including some component of aeroelastic
            coupling. ``Considering aeroelastic effects is essential when
            simulating multi-megawatt turbines; the assumption of small blade
            deformations...loses accuracy with increasing blade size.'' (pg.15).
            Keyword: FAST for the coupled aeroelastic model.
        \item Advocates for AD+R approaches to modeling turbine rotor as a good
            balance of cost-effectiveness for modeling far wakes and wake
            interactions. Probably the regime we're in for power production modeling.
        \item Notes the importance of the use of a ``power controller to actively determine
            the rotational speed of the rotors.''
        \item There is no definitive best approach between synthetic and precursor LES modeling
            for the ABL.
    \end{itemize}
\end{itemize}

\subsection{LES Implementations}

\longcite{deskosWInc3DNovelFramework2020}

\begin{itemize}
    \item \textbf{Overview:}
        % \item \textbf{Strong points:} 
    %     \begin{itemize}
    %     \end{itemize}
    % \item \textbf{Weak points:} 
    % \begin{itemize}
    %     \item 
    % \end{itemize}
    \item \textbf{Details:} 
    \begin{itemize}
        \item Finite-difference discretization.
        \item Uses the Smagorinsky SGS model, with some Mason-Thomson correction (not sure
            if this is standard or not).
    \end{itemize}
    \item \textbf{Claims:}
        % \begin{itemize}
        % \item Notes that industry primarily uses analytical, "reduced-order wake models"
        %     to estimate annual energy production.
        %     Cites \cite{nygaardLargescaleBenchmarkingWake2022} to argue that
        %     these models struggle with reproducing wind farm blockage and
        %     farm-farm wake interactions.
        % \item Notes that "only a few" existing LES implementations can tackle
        %     ``gravity wave effects.''
        % \item They argue that finite-volume approaches, by virtue of allowing for ``grid stretching,''
        %     enable the resolution of larger domains with the same number of degrees of freedom
        %     and also providing more ``geometric flexibility.''
        % \end{itemize}
\end{itemize}

\longcite{stipaTOSCAOpensourceFinitevolume2024}

\begin{itemize}
    \item \textbf{Overview:} Open-source, finite-volume LES aimed at
        large-scale studies of wind farm-induced gravity waves and the
        interaction between cluster wakes and the atmosphere. Specifically
        designed for LES of large finite wind farms.
    % \item \textbf{Strong points:} 
    %     \begin{itemize}
    %     \end{itemize}
    % \item \textbf{Weak points:} 
    % \begin{itemize}
    %     \item 
    % \end{itemize}
    \item \textbf{Details:} 
    \begin{itemize}
        \item Supports actuator line, ALM, actuator disk, ADM, and uniform
            actuator disk, UAD methods. Presumably UAD is AD-NR, and ADM is
            AD-R.
        \item Finite-volume framework.
        \item Supports a ``concurrent-precursor method,'' which tye argue may
            not be available in other finite-volume solvers (even though it is
            ``extensively used in pseudo-spectral methods.'').
        \item Supports a ``sharp-interface immersed boundary method (IBM)''
            that they argue allows for the simulation of moving objects and
            complex terrain features.
        \item Enforces a ``desired hub-height wind speed'' while avoiding ``inertial oscillations
            produced by the Coriolis force above the boundary layer.'' No idea what this means.
    \end{itemize}
    \item \textbf{Claims:}
        \begin{itemize}
        \item Notes that industry primarily uses analytical, "reduced-order wake models"
            to estimate annual energy production.
            Cites \cite{nygaardLargescaleBenchmarkingWake2022} to argue that
            these models struggle with reproducing wind farm blockage and
            farm-farm wake interactions.
        \item Notes that "only a few" existing LES implementations can tackle
            ``gravity wave effects.''
        \item They argue that finite-volume approaches, by virtue of allowing for ``grid stretching,''
            enable the resolution of larger domains with the same number of degrees of freedom
            and also providing more ``geometric flexibility.''
        \end{itemize}
\end{itemize}


\newpage

\printbibliography

\end{document}

%%% Local Variables:
%%% mode: latex
%%% TeX-master: t
%%% End:
% fonts
